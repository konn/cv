\documentclass[letterpaper]{scrartcl}
\usepackage{mymacros}
\usepackage{graphicx}
\usepackage{enumitem}
\DeclareMathAlphabet{\mathrsfs}{U}{rsfso}{m}{n}
\renewcommand{\mathscr}[1]{\mathup{\mathrsfs{#1}}}
\usepackage[super]{nth}
\usepackage[bookmarksnumbered,pdfproducer={LuaLaTeX},%
            luatex,psdextra,pdfusetitle,pdfencoding=ydnt]{hyperref}
\usepackage[backend=biber,style=math-numeric,sorting=none]{biblatex}
\addbibresource{myreference.bib}
\newcommand{\leport}{$\lambda$eport}
\setlist{itemsep=.25ex}

\usepackage{url}	% required for `\url' (yatex added)
\usepackage{amssymb}	% required for `\mathbb' (yatex added)
\usepackage{multicol}	% required for `\multicols' (yatex added)
\begin{document}
\begin{center}
 {\huge \bfseries Hiromi ISHII}
  \vskip 1em
 Research and Development Department, DeepFlow, Inc.\\
 Visiting Researcher at Institute of Statistical Mathematics.\\
 \textsc{E-Mail}: \href{mailto:konn.jinro@gmail.com}{\nolinkurl{konn.jinro@gmail.com}}\\
 \textsc{Web Site}: \url{https://konn-san.com}\\
 \textsc{GitHub}: \url{https://github.com/konn}
\end{center}

 \section*{Education}
 \begin{tabular}[t]{@{}p{.2\linewidth}p{.75\linewidth}@{}}
 \begin{minipage}[t]{\linewidth}Ph.D. in Math\\ 2016 - 2019\end{minipage} 
 & \begin{minipage}[t]{\linewidth}
   University of Tsukuba, Tsukuba-city, Ibaraki, Japan.\\
   \emph{Thesis:} Bidirectional Interplay between Mathematics and Computer Science: Safety and Extensibility in Computer Algebra and Haskell \cite{Ishii:2019dp}.
 \end{minipage}
 \\[4.5em]
 \begin{minipage}[t]{\linewidth}M.S. in Math\\ 2016\end{minipage} 
 & \begin{minipage}[t]{\linewidth}
    University of Tsukuba, Tsukuba-city, Ibaraki, Japan.\\
   \emph{Thesis:} On Regularity Properties of Sets of Reals and Inaccessible Cardinals \cite{ISHII:2016sf}.
   \end{minipage} 
 \\[3em]
   B.A, 2014 & \begin{minipage}[t]{\linewidth}
    Waseda University, Tokyo, Japan, 2014,\\
    \emph{Summa cum laude}.
    \end{minipage}
  \\
 \end{tabular}

 \section*{Job Career}
 \begin{description}
  
 \item[Apr 2019--Present]
    Visiting Researcher at Institute of Statistical Mathematics.
 \item[Apr 2019--Present]
    Full-time Researcher and Developer at Research and Development Department, DeepFlow, Inc.
 \item[Jan 2019--Mar 2019]
   Part-time developer at DeepFlow, Inc.
 \item[Apr 2018--Mar 2019]
            Part-time developer at Clear Code, Inc.
 \item[2014] Google Summer of Code 2014. Theme: ``An Efficient Computational Algebra and Symbolic Linear Algebra Library in Haskell''
 \item[Apr 2014--Mar 2017] Part-time Job of System Administrator at College of Mathematics, Tsukuba University
 \item[Oct 2010--Mar 2014] Part-time developer in Preferred Infrastructure, Inc.
 \item[Aug--Sep 2010] Internship student in Preferred Infrastructure, Inc.
 \end{description}

 \section*{Research Experiences}
 \begin{description}
  \item[April 2018--]
    Joint research on discrete mathematics with Momoko Hayamizu  (the Institute of Statistical Mathematics).
  \item[April 2017--March 2019]
    Research Fellowship for Young Scientists (DC2), Japan Society for the Promotion of Science (2017-2019).

    Theme: ``Properties of sets of reals and its applications to Computer''
  \item[2014--2017] Teaching Assistant of ``Programming Exercise'', University of Tsukuba.
 \end{description}

 \section*{Selected Fellowships, Awards and Special Note}
 \begin{itemize}
 \item Research Fellowship for Young Scientists (DC2), Japan Society for the Promotion of Science (2017-2019)
 \item 14th Meikei Prize (2016)
 \item Highest Award of Deans' Prize of School of Fundamental Science and Engineering, Waseda University (2014)
 \item A committer of \href{https://github.com/Khan/KaTeX}{KaTeX}, a Web Math Renderer, and \href{https://www.yesodweb.com}{Yesod}, a Web Framework.
 \end{itemize}

 \section*{Research}
 Main research area is mathematics, especially mathematical logic (especially axiomatic set theory) and computational algebra.
 My main interests are set-theoretic properties of sets of reals and their computational application; the related area includes Gr\"{o}bner basis computation, functional programming, theoretical computer science and discrete mathematics.

 \subsection*{Conference Presentations}
 \begin{itemize}
  \item December 2018, \emph{A Purely Functional Computer Algebra System Embedded in Haskell}, Computer Algebra -- Theory and its Applications 2018, Kyoto, Japan, December 17-20.
  \item September 2018, \emph{A Purely Functional Computer Algebra System Embedded in Haskell}, Computer Algebra in Scientific Computing (CASC) 2018, Lille, France, September 17-21. (Refereed)
  \item November 2017, \emph{Reflection Principle and construction of saturated ideals on $\Pow_{\omega_1} \lambda$}. Workshop on Iterated Forcing Theory and Cardinal Invariants, Kyoto, Japan, (Non-refereed).
  \item March 2016, \emph{Freer Monads, More Extensible Effects}. Programming and Programming Language Workshop (PPL) 2016, Okayama, Japan (Refereed).
 \end{itemize} 
\subsection*{Papers}
\nocite{Ishii:2019dp,ISHII:2018ek,ISHII:2016sf,Kiselyov:2015xy}
\printbibliography[heading=none]

\section*{Selected Development Examples}
\vspace{-1em}
\noindent \textbf{Main Language}: \emph{Haskell}, \\
\noindent \textbf{Experienced Languages}: \emph{JavaScript, Objective-C, Ruby, Lean, Agda, Rust, Scala}.
\vspace{-.5em}
\subsection*{\leport, an automatic report rating system}
 \vspace{-.5em}\noindent \textbf{Keywords}: \emph{Property-based Testing, Let's Encrypt, Distributed Programming, Linux Containers, EventSrouce.}

 \vspace{.5em}
 At the University of Tsukuba, we teach Haskell to students as an exercise.
 The \leport{} provides an automatic report rating service and is mainly used by teaching assistants.

 Applying the method of \emph{Property-based Testing} in the realm of functional programming, it accepts formal specifications of exercises and automatically evaluate students' answers using \emph{QuickCheck} library in Haskell.

 Since \leport{} treats privacy-sensitive information, we use HTTPs and use the TLS certificate issued by \emph{Let's Encrypt}.
 \leport{} makes use of \emph{Cloud Haskell}, a distributed programming framework, and adopting \emph{Work-stealing model} to build distributed rating system, enabling multiple users to rate reports simultaneously.
 It also utilises \emph{Software Transactional Memory} of Haskell to achieve efficient concurrency.
 To prevent malicious program from being executed and control the resource usage, it compiles and runs the report source-codes in isolated containers provided by \emph{Linux Containers}.
 To make rating process seamless, rating results will be communicated to web frontend per-exercise.
 Linux Container's \emph{WebSocket} interface for inter-process communication and \emph{EventSource} technology made this possible.

 \subsection*{Computer Algebra System \texttt{computational-algebra}}
 \vspace{-.5em}\noindent
 \textsc{Web Site}: \url{https://konn.github.io/computational-algebra}\newline
 \textbf{Keywords}: \emph{Google Summer of Code 2014, CASC 2018, Computer Algebra, Type-level programming, Parallel Computation, Formal Method.}

 \vspace{.5em}
 A safe and reliable computational algebra system implemented as an EDSL in Haskell, powered by the progressive type-system of Glasgow Haskell Compiler.
 It enjoys the following advantages:
 \begin{description}
  \item[Type-Safety] Encoding fields and arity as a type, it prevents wrong operations;
  \item[Extensibility] Providing type-classes for polynomials and algebraic systems, users can enjoy various algorithms regardless to implementation detail;
  \item[Intuitive] The library automatically gives the injection mapping between two distinct rings such as $\Q[x, y, z]$ and $\Q[x,z,y, w]$, based on their types.
             One can also write polynomials much alike to ordinary Math, for example: \verb|#x ^ 2 + #z * #x - 2|.
  \item[Static Verification] 
     Using QuickCheck, correctness of algorithms are statically verified.
 \end{description}

 In \emph{Google Summer of Code 2014}, I tried to implement two algorithms $F_4$ and $F_5$, which are known to be the fastest state-of-the-art algorithms for Gr\"{o}bner basis computation.
 To that end, I implemented a framework to treat multiple matrix libraries uniformly.
 Also, other algorithms, such as polynomial factorisation and algebraic real computation, are provided.
 Large part of these implementation is verified with the method of property-based testing.

 To achieve type-safety in a full spectrum, I implemented two libraries: the \href{http://hackage.haskell.org/package/type-natural}{\texttt{type-natural}} package and the \href{http://hackage.haskell.org/package/ghc-typelits-presburger}{\texttt{ghc-typelits-presburger}} compiler plugin.
 The former provides the abstraction over various type-level naturals, and the latter augments the GHC's type-checker with Presburger Arithmetic solver.
 Based on them, I implemented the \href{http://hackage.haskell.org/package/sized}{\texttt{sized}} package, which provides a functionality to turn arbitrary sequential data-types into fixed-size container, indexed with type-level naturals, used for efficient representation of monomials.

 As a teaching assistant, I supervised the graduation work using this library in Mathematics Department in University of Tsukuba.
 For more details, see the recent paper \cite{ISHII:2018ek}.

 \subsection*{Other Development Activity}
 \begin{itemize}
 \item I contributed to \emph{\href{https://khan.github.io/KaTeX/}{\emph{KaTeX}}}, a Open-source Web Math Renderer, and become a committer.
       Although I'd never experienced modern JavaScript development, I could caught up with it within half a day.
 \item As a part-time researcher and developer at Clear Code Inc, I am currently working on mathematical R\&D project which adopts the progressive workflow with Docker, GitLab CI and Property-based testing.
 \item I'm the original author of OAuth library for \emph{\href{https://www.yesodweb.com}{Yesod}}, a Haskell Web Framework.
 \item As a pre-publication reviewer, I contributed to Japanese translation of famous textbooks such as ``Types and Programming Languages'' (B. Pierce) and ``Parallel and Concurrent Programming in Haskell'' (S. Marlow).
 \item See GitHub (\url{https://github.com/konn}) for more about my development activity.
 \end{itemize}
\end{document}

%  LocalWords:  summa laude acm sigplan
