\RequirePackage{luatex85}
\documentclass[a4j,draft]{ltjsarticle}
\usepackage{mymacros}
\usepackage{graphicx}
\usepackage[hiragino-pro]{luatexja-preset}
\usepackage{luatexja-ruby}
\usepackage{luatexja-otf}
\usepackage{fixme}
\DeclareMathAlphabet{\mathrsfs}{U}{rsfso}{m}{n}
\renewcommand{\mathscr}[1]{\mathup{\mathrsfs{#1}}}
\usepackage[super]{nth}
\usepackage[bookmarksnumbered,pdfproducer={LuaLaTeX},%
            luatex,psdextra,pdfusetitle,pdfencoding=auto]{hyperref}
\usepackage[backend=biber,style=math-numeric,sorting=none]{biblatex}
\addbibresource{cv-references.bib}
\renewcommand{\emph}[1]{\textbf{\textgt{#1}}}

\usepackage{url}	% required for `\url' (yatex added)
\usepackage{amssymb}	% required for `\mathbb' (yatex added)
\begin{document}
\begin{center}
 {\huge \bfseries \ruby{石}{いし}\ruby{井}{い} \ruby{大海}{ひろみ}}
 \vskip 1em
 {\emph{\Large 基本情報}}\\
 DeepFlow 株式会社\\
 \textsc{E-Mail}: \href{mailto:konn.jinro@gmail.com}{\nolinkurl{konn.jinro@gmail.com}}\\
 \textsc{Web Site}: \url{https://konn-san.com}\\
 \textsc{GitHub}: \url{https://github.com/konn}\\
 \textsc{Zenn}: \url{https://zenn.dev/konn}
\end{center}

\section*{教育・学位}
\begin{tabular}[t]{@{}p{.2\linewidth}p{.75\linewidth}@{}}
 \begin{minipage}[t]{\linewidth}\emph{博士(理学)}\end{minipage} 
 & \begin{minipage}[t]{\linewidth}
    2019年3月.筑波大学数理物質科学研究科数学専攻 博士後期課程・修了\\
    \emph{博士論文:} Bidirectional Interplay between Mathematics and Computer Science: Safety and Extensibility in Computer Algebra and Haskell
  \end{minipage}\\
 \begin{minipage}[t]{\linewidth}\emph{修士(理学)}\end{minipage} 
 & \begin{minipage}[t]{\linewidth}
    2016年3月.筑波大学数理物質科学研究科 博士前期課程・修了\\
   \emph{修士論文:} On Regularity Properties of Sets of Reals and Inaccessible Cardinals
   \end{minipage} 
 \\
 \begin{minipage}[t]{\linewidth}\emph{学士(理学)}\end{minipage} 
 & 2014年3月.早稲田大学基幹理工学部数学科・卒 \\
\end{tabular}

\section*{職歴・採用歴}
\begin{refsection}
  \newrefcontext[labelprefix={J-}] 
\begin{description}
 \item[2019年4月-現在] DeepFlow株式会社 研究開発部.
 
  Haskellによる大規模並列数値計算ソルバーElkurageの設計・開発に従事.当該ソルバーはスレッド並列やMPI通信などを活かしたHaskellによる高性能プログラミングの技術を用い,数億セルの並列計算を実現.高い抽象性と拡張性・効率性を両立するためのHaskell内への埋め込みドメイン特化型言語の設計や,型システムの設計等を主に担当する.また,研究・産業用途のシミュレーションワークフロー・最適化ツールを,顧客との綿密なコミュニケーションを通じて設計・開発.

  主な実績は以下の通り:
  \begin{itemize}
    \item 大規模数値計算ソルバ用の依存型つきHaskell製DSLの設計・実装
    \begin{itemize}
      \item 個別の方程式系や,ソルバのプラグイン機構などを適切な依存型を用いて代数的に抽象化・実装
      \item 任意カインドでラベル付け可能な拡張可能レコードを実装し,型制約の自動解消を行う型検査器プラグインも併せて開発
      \item 詳細については,Haskell Day 2019での発表資料~\cite{ISHII:2019hd-ja}やプレプリント~\cite{ISHII:2021wt}を参照.
    \end{itemize}
    \item メッシュ生成ライブラリGmshへのHaskellバインディングを,Higher-Kinded Datatypesの技法などを用いて実装
    \item 産業・学術用シミュレーションGitOpsワークフローツールの開発
      \begin{itemize}
      \item 単一の設定ファイルから必要な設定・入力をパラメトリックに生成
      \item クラスタにインストール済のジョブマネージャまたはスレッド並列・Cloud Haskellなどを用いた独自実装のクエジューラを通じて並列で多数のジョブを同時実行
      \item ParaViewのPythonスクリプティング機能による自動可視化
      \item 各ワークフローのジョブ間の複雑な依存関係はShakeの規則により記述
      \item 実行結果のレポートや比較表をHTMLなどとしてHTMLやPDFで出力,R2やローカルストレージ上に保存しServantサーバよりサーブ
      \end{itemize}
  \end{itemize}
 \item[2019年4月-2020年3月] 統計数理研究所 外来研究員
 \item[2017年4月-2019年3月]
              日本学術振興会特別研究員 DC2

              研究課題:『実数の集合の性質の集合論的解明と工学的応用』
 \item[2014年-2017年] 筑波大学数学類『計算機演習』ティーチング・アシスタント
 \item[2014年4月] Google Summer of Code 2014採択

            題目:『Haskellによる効率的なGr\"{o}bner基底計算とそのための疎行列対角化アルゴリズムの実装』
 \item[2014年4月-2017年3月] 筑波大学数学専攻計算機管理アルバイト(www-admin)
 \item[2010年10月-2014年3月] 株式会社Preferred Infrastructureアルバイト
 \item[2010年8月-9月] 株式会社Preferred Infrastructureインターン生
\end{description}
\printbibliography[title={職歴欄参考文献},heading=subbibliography]
\end{refsection}
\section*{賞罰・特記事項}
\begin{itemize}
 \item 日本学術振興会特別研究員DC2,(2017年〜2019年)
 \item 第十四回茗渓会賞(2016年)
 \item 2013年度早稲田大学基幹理工学部長賞最優秀賞(第一回),基幹理工学部卒業生総代
\end{itemize}

\section*{研究業績}

\subsection*{査読付き会議論文}
\noindent
August 2021, \emph{Automatic Differentiation with Higher Infinitesimals, or Computational Smooth Infinitesimal Analysis in Weil Algebra}. Computer Algebra in Scientific Computing 2021, Sochi, Russia.

\noindent
September 2018, \emph{A Purely Functional Computer Algebra System Embedded in Haskell}. Computer Algebra in Scientific Computing 2018, Lille, France.

\noindent
September 2015. Oleg Kiselyov and Hiromi ISHII, \emph{Freer Monads, More Extensible effects}. Haskell Symposium 2015, Vancouber, Canada.

\subsection*{学会発表}
\noindent
August 2021, \emph{Automatic Differentiation with Higher Infinitesimals, or Computational Smooth Infinitesimal Analysis in Weil Algebra}. Computer Algebra in Scientific Computing 2021, Sochi, Russia, 査読あり.

\noindent
September 2018, \emph{A Purely Functional Computer Algebra System Embedded in Haskell}. Computer Algebra in Scientific Computing 2018, Lille, France, 査読あり.

\noindent
March 2016, \emph{Freer Monads, More Extensible Effects}. Programming and Programming Language Workshop (PPL) 2016, Okayama-prefecture, Japan, 査読あり.

\noindent
November 2017, \emph{Reflection Principle and construction of saturated ideals on $\Pow_{\omega_1} \lambda$}. Workshop on Iterated Forcing Theory and Cardinal Invariants, Kyoto-prefecture, Japan, 査読なし.

\begin{refsection}
  \nocite{ISHII:2016sf,ISHII:2018ek,ISHII:2019aa,Ishii:2021tp,Kiselyov:2015xy}
  \defbibnote{exteff}{特に、Oleg Kiselyov氏との共著論文``Freer Monads, More Extensible Effects''~\cite{Kiselyov:2015xy}は現代のイフェクトシステムの隆盛につながるもので、特筆に値する。}
 \printbibliography[title=論文,heading=subbibliography,prenote=exteff]
\end{refsection}

\section*{自然言語能力}
\begin{description}
  \item[日本語] 母語
  \item[英語] 流暢(参考記録:TOEIC 920点(2012年))
\end{description}

\section*{プログラミング言語能力}

\begin{description}
  \item[Haskell] エキスパート。公私ともに主要利用言語。
  \item[Python] 中級(機械学習案件や可視化に利用経験あり。環境管理にはRyeを使用)
  \item[Agda] 趣味(代数的性質の証明や、Cubical Type Theoryの利用経験あり)
  \item[JavaScript / TypeScript] 中級(一部自動化やGitHub Actionsなどに)
  \item[Rust] 初心者(非常に興味あり)
  \item[C] 学部卒程度(HaskellのFFIバインディングライブラリを開発する程度の技能はあり)
\end{description}

\section*{技術要素}
以下の技術要素に熟達している:

\begin{description}
  \item[並列・分散処理] スレッド並列、STM、\texttt{async}型ライブラリ、Cloud Haskell、Open MPI
  \item[ストリーム処理] Haskellにおいては\texttt{streaming}ライブラリーを利用。\texttt{conduit}系統の利用経験も長い。
  \item[EDSLの設計] ドメインに応じたHaskell内DSLの設計・実装経験豊富。
  \item[線型型] リソース割り当てや管理などに利用。
  \item[コンテナ技術] Docker, Singularity
  \item[CI/CD] GitHub Actions, GitLab CI/CD
  \item[形式手法] 性質ベーステスト(QuickCheck, Falsify, hypothesis)リグレッションテスト、依存型によるコンパイル時検証
  \item[クラウド技術] AWS, GCP, Azure, さくらのクラウド, Terraform, Packer
\end{description}

\section*{オープンソース開発実績(一部)}
\begin{refsection}
\newrefcontext[labelprefix={OSS-}] 

\subsection*{モノレポ依存関係管理ツールGuardian}
\vspace{-1em}
\noindent
\emph{URL}: \url{https://github.com/deepflowinc/guardian}
\vspace{1em}

DeepFlowのプロダクトは非常に多くの内製パッケージから成っており,それらの間の依存関係が複雑化することでビルド時間に大きな悪影響が出ていた.
こうした状況を改善するため,パッケージを複数のグループに分類し,グループ間の依存関係に関する制約を設定させCIで絶えずチェックさせることで,依存関係において疎結合・関心の分離を実現するためのツールGuardianの開発を社内で主導し,OSSとして公開した.
ツールの詳細についてはZenn上のブログ記事~\cite{ISHII:2023gd}を参照.

\subsection*{Haskell用Cloudflare Workerライブラリ}
\vspace{-1em}
\noindent
\emph{URL}: \url{https://github.com/konn/ghc-wasm-earthly}
\vspace{1em}

GHC 9.10より実装された,WASMバックエンドのJavaScript FFI機能を用いて,サーバレスウェブアプリ開発環境Cloudflare Worker上のアプリをHaskellで開発するためのライブラリを実装している.
開発の途上で得られた知見は,Zenn上のブログ記事~\cite{ISHII:2024cf}として公開済である.
これは以下の複数のパッケージからなる:

\begin{itemize}
  \item \texttt{ghc-wasm-jsobjects}: 型がついていない\verb|JSVal|を\verb|newtype|で包み,JavaScriptのオブジェクト階層を模倣することを企図したHaskellライブラリ.
  \item \texttt{webidl-codegen-wasm}: WebIDLによるJavaScript APIの記述から,上記の\texttt{ghc-wasm-jsobjects}での利用を念頭に置いたGHCのWASMバックエンドのJavaScript FFI用の型・ブリッジ関数を生成するツール.
  \item \texttt{cloudflare-worker}: Cloudflare WorkerのランタイムAPIへのバインディング.
\end{itemize}

これらのライブラリを用いてServantをWorkers上に移植したり,線型型を用いたよりリソース安全な定式化を追求するのが目下の目標である.

\subsection*{Haskell向け依存型プログラミング用ライブラリ群}
Haskellでの依存型を用いたプログラミングを支援するためのライブラリ群をいくつか開発し,業務においても利用されている.

\begin{itemize}
  \item \texttt{ghc-typelits-presburger}~\cite{ghc-typelits-presburger}: Presburger算術(整数計画法の理論)の範囲の型レベル等式・不等式制約を自動的に解消する型検査器プラグイン。Singletons版もあり。
  \item \texttt{type-natural}~\cite{type-natural}: GHCの型レベル自然数に対するsingletonと種々の性質の証明を提供するライブラリ。
  \item \texttt{sized}~\cite{sized}: ベクトルなどに型レベルで静的に長さをつけて扱うためのライブラリ。
\end{itemize}

\subsection*{Linear Haskell向けユーティリティライブラリ}
\vspace{-1em}
\noindent
\emph{URL}: \url{https://github.com/konn/linear-extra}
\vspace{1em}

Linear Haskellを実用する上で便利な機能のうち、\texttt{linear-base}に欠けているものを提供するライブラリ。参照カウンタや任意ベクトルの線型割り当て機能、Linear Constraint導入までの代替手段である証拠ベースの資源割り当て機能などを提供する。
Haskellにおける線型型については、Zenn記事~\cite{ISHII:2023lh1,ISHII:2023lh2}の執筆を通して日本語圏での普及に務めている。

\subsection*{Haskell Language Serverのメンテナー}
Haskell の公式言語サーバ(IDEバックエンド)であるHaskell Language Serverのメンテナの一人を務める.主な貢献事項は以下の通り:
  \begin{enumerate}
    \item Splice Plugin: Template Haskellマクロを評価し,その結果でマクロ呼び出しを置換するためのプラグインを実装した.
    \item Disambiguate Imports: 複数モジュールまたは現在のモジュールで定義され衝突が発生している曖昧な識別子を一意化するための機構を提案し,実装した.
    \item Eval Pluginの改良: パーザの効率改善,\texttt{:type}や\text{:kind}コマンドの実装などを行った.
  \end{enumerate}

\subsection*{計算機代数システム\texttt{computational-algebra}}
\vspace{-1em}
\noindent
 \textsc{Web Site}: \url{https://konn.github.io/computational-algebra}
 \vspace{1em}

 関数型言語Haskellの処理系Glasgow Haskell Compiler(以下,GHC)が提供する先進的型システムを活かし,高度な代数計算を安全に行える計算代数システムをEDSLとしてゼロから設計・実装した.
 具体的には,型システムや形式手法を応用することで以下の長所を実現した:
 \begin{description}
  \item[型安全] 係数体や変数の数,順番などを型で表すことで,誤操作を防止する.
  \item[拡張性] 多項式や代数系を表す型クラスを提供し,内部実装に依らず様々なアルゴリズムを適用可能にした.
  \item[直感的] $\Q[x, y, z]$と$\Q[x,z,y, w]$は型上区別されるが,これらの間の埋め込み写像を型情報だけから自動的に計算出来る.
             また,多項式も本来の数式に近い形で \verb|#x ^ 2 + #z * #x - 2| のように書ける.
  \item[静的保証] 
     QuickCheckを用いて\emph{アルゴリズムの形式仕様を静的に検証}し,ライブラリの品質を保っている.
 \end{description}

 上述の\emph{Google Summer of Code 2014}では,高速なGr\"{o}bner基底計算アルゴリズムとして知られる$F_4$および$F_5$アルゴリズムの実装を試み,複数の行列ライブラリを統一的に扱うための枠組みや,ライブラリの整備を進めた.
 その他に,有理係数多項式の因数分解や,代数的数の計算機能なども実装されている.

 特に,型安全性を最大限実現するために,GHCの型レベル自然数機能を強化する\href{http://hackage.haskell.org/package/type-natural}{\texttt{type-natural}}パッケージや,GHCの型検査器にPresburger算術ソルバを組込むコンパイラプラグイン\href{http://hackage.haskell.org/package/ghc-typelits-presburger}{\texttt{ghc-typelits-presburger}}を開発した.
 それらを用い,リストや配列,ベクトルなど任意のシーケンシャルなデータ型から固定長のコンテナ型を創り出せる\href{http://hackage.haskell.org/package/sized}{\texttt{sized}}パッケージを実装した.

 本ライブラリを用いた筑波大学数学類の卒業研究をティーチング・アシスタントとして指導した.
 また,エクアドル大学のヤチャイ技術大学の研究プロジェクト等でも用いられているようである.

\subsection*{その他の開発活動}
\begin{itemize}
 \item Web上で高速かつ高品質な数式描画を可能とする\href{https://khan.github.io/KaTeX/}{\emph{KaTeX}}に幾つかのコマンド・環境を実装し,コミッタ権限を得た.
       これまで本格的なJavaScript開発は経験していなかったが,\texttt{npm}や\texttt{browserify},\texttt{flow}などを用いたワークフローには半日程度でキャッチアップ出来た.
 \item Haskellの\emph{Webフレームワーク\href{https://www.yesodweb.com}{Yesod}}のOAuth認証ライブラリの開発(現在は引退).
 \item 『型システム入門 ─プログラミング言語と型の理論─』『Haskellによる並列・並行プログラミング』(ともにオーム社)など定評のある技術書・学術書の邦訳に,出版前レビュワーとして参画・貢献した.
 \item その他の開発プロジェクトは,GitHub上(\url{https://github.com/konn})にて見ることが出来る.
\end{itemize}

\printbibliography[title={OSS参考文献},heading=subbibliography,sorting=none]
\end{refsection}
\end{document}
